Lysdioder har forskjellige spenningsfall for samme farge avhengig av modellserie og produsent. For optimal verdi må man lese databladet til dioden. 

Tabell \ref{tab:LEDspenningsfall} viser et generelt spenn av verdier.

\begin{table}[!ht]
	\caption{Spenningsfall for forskjellige lysdioder}
	\label{tab:LEDspenningsfall}
	\begin{center}
		\begin{tabular}{|l|c|c|} 
			\hline
			Farge & Spenningsfall & Enhet \\ [0.5ex] 
			\hline\hline
			Hvit & $3,0 - 5,0$ &$ [V]$ \\
			\hline
			Fiolett: &  $2,8 - 4,0$ &$ [V]$\\
			\hline
			Blå: &  $2,5 - 3,7$ &$ [V]$\\
			\hline
			Grønn: &  $1,6 - 4,0$ &$ [V]$\\
			\hline
			Gul: &  $2,0 - 2,4$ &$ [V]$\\
			\hline
			Oransje: &  $2,0 - 2,1$ &$ [V]$\\
			\hline
			Rød: &  $1,5 - 2,0$ &$ [V]$\\
			\hline
			Infrarød: &  $1,2 - 1,9$ &$ [V]$\\
			\hline
		\end{tabular}
	\end{center}
\end{table}

