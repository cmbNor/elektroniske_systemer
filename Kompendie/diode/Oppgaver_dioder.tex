
\begin{question}[name=Oppgave, topic=dioder]
	Tegn symbolene for følgende komponenter.
	\begin{enumerate}[label=\roman*)]
		\item Halvlederdiode
		\item LED
		\item Zenerdiode
	\end{enumerate}
\end{question}

\vspace{0.5cm} % Add space after the solution

\begin{solution}[name=Løsningsforslag oppgave]
Symboler vist i Figur \ref{fig:diodeSymb}.
	\begin{figure}[H]
		\centering
		\includegraphics[width=0.7\textwidth]{diode/figurer/oppgave1.png}
		\caption{Eksempel på forskjellige symboler for dioder.}
		\label{fig:diodeSymb}
	\end{figure}
\end{solution}

\vspace{0.5cm} % Add space after the solution

\begin{question}[name=Oppgave, topic=dioder]
Hva betyr de følgende begrepene i sammenheng med dioder? Svar på spørsmålet og tegn eksempel.
	\begin{enumerate}[label=\roman*)]
		\item Lederetning
		\item Sperreretning
		\item Anode
		\item Katode
		\item Zenerspenning
	\end{enumerate}
\end{question}

\vspace{0.5cm} % Add space after the solution

\begin{solution}[name=Løsningsforslag oppgave]
	\begin{enumerate}[label=\roman*)]
		\item Diode koblet slik at den leder strøm
			\begin{figure}[H]
			\centering
			\includegraphics[width=0.4\textwidth]{diode/figurer/lederetning.png}
			\caption{Diode koblet i lederetning}
			\label{fig:lede}
		\end{figure}
		\item Diode koblet slik at den ikke leder strøm
			\begin{figure}[H]
				\centering
				\includegraphics[width=0.4\textwidth]{diode/figurer/sperreretning.png}
				\caption{Diode koblet i sperreretning}
				\label{fig:sperre}
			\end{figure}
		\item Anode er siden så hvis man kobler den til den positive siden av en kilde, så vil dioden lede. Strømmen går fra anode til katode

			\begin{figure}[H]
				\centering
				\includegraphics[width=0.4\textwidth]{diode/figurer/anodekatode.png}
				\caption{Anode og katode på diode}
				\label{fig:anoKato}
			\end{figure}
		\item Dersom dioden leder er katoden koblet til den negative siden av kilden som vist i Figur \ref{fig:anoKato}
		\item Zenerspenning er spenningen hvor en zenerdiode begynner å lede strøm i sperreretning, og kan stabiliserer spenningen i kretsen.
	\end{enumerate}
\end{solution}

\vspace{0.5cm} % Add space after the solution

\begin{question}[name=Oppgave, topic=dioder]
	Beskriv tre bruksområder for en halvlederdiode.
\end{question}

\vspace{0.5cm} % Add space after the solution

\begin{solution}[name=Løsningsforslag oppgave]
	\begin{enumerate}[label=\roman*)]
		\item Sperre for strøm i én retning
		\item For å beskytte transistorer og andre følsomme komponenter, kobles dioden som en friløpsdiode når den brukes sammen med en induktiv last.
		\item Likerette AC til DC
	\end{enumerate}
\end{solution}

\vspace{0.5cm} % Add space after the solution



% LED


\begin{question}[name=Oppgave, topic=LEDdioder]
	En LED har et spenningsfall i lederetning på $2,5 [V]$ og det kreves en strøm på $15 [mA]$ for at den skal lyse. Den tilkoblede spenningskilden har en spenning ut på $15 [V]$.

	Beregn størrelsen på seriemotstanden til dioden.
\end{question}

\vspace{0.5cm} % Add space after the solution

\begin{solution}[name=Løsningsforslag oppgave]
Først finner vi spenningsfallet vi må ha over motstanden for at dioden skal ha et spenningsfall på $2,5 [V]$.


\[U_{R-serie}=U_{Kilde}-U_{LED}=15-2,5=12,5 [V]\]

Finner størrelsen på resistansen som sørger for maksimal strøm på $15 [mA]$.

\[R_{Serie}=\frac{U_{R-serie}}{I_{LED}}=\frac{12,5}{15\cdot10^{-3}}=830 [\Omega]\]

\end{solution}


\begin{question}[name=Oppgave, topic=LEDdioder]
Diodene $D_1,D_2$ og $D_3$ er helt identiske og koblet i parallell som vist i Figur \ref{fig:LEDpara}. Spenningsfallet over diodene i lederetning skal være $2,5[V]$ og strømmen skal være $14 [mA]$.


\begin{enumerate}[label=\roman*)]
	\item Finn verdien for $R_1, R_2$ og $R_3$.
	\item Benytt Tabell \ref{tab:LEDspenningsfall} og finn ut hvilken farge diodene mest sannsynlig har.
\end{enumerate}


	\begin{figure}[H]
	\centering
	\includegraphics[width=0.5\textwidth]{diode/figurer/LEDparallell.png}
	\caption{LED i parallell}
	\label{fig:LEDpara}
\end{figure}
\end{question}

\vspace{0.5cm} % Add space after the solution

\begin{solution}[name=Løsningsforslag oppgave]
	\begin{enumerate}[label=\roman*)]
		\item Siden alle de tre parallelle grenene er identiske kan vi gjøre beregninger på én av de siden spenningen og den nominelle strømmen er lik for alle grenene.

		Velger grenen nærmest kilden og finner ønsket spenningsfallet over motstanden.
		\[U_{R_{1}}=U_1-U_{D_{1}}=10-2,5=7,5 [V]\]

		Finner motstandsverdien som sørger for at spenningsfallet blir $7,5 [V]$

		\[R_1=\frac{U_{R_{1}}}{I_{D_{1}}}=\frac{7,5}{14 \cdot 10^{-3}} \approx 536 [\Omega] \]
		\item Dioden har mest sannsynlig fargen blå.
	\end{enumerate}

\end{solution}
\vspace{0.5cm} % Add space after the solution


\begin{question}[name=Oppgave, topic=dioder]
Beregn seriemotstandverdien for en LED basert på databladet presentert i Vedlegg \ref{paper-a}. Dioden skal drives av en kilde som har følgende spenning $U_{kilde}=9 [V]$.
\end{question}

\vspace{0.5cm} % Add space after the solution

\begin{solution}[name=Løsningsforslag oppgave]
Leser av forward voltage $V_f=2 [V]$ @ $I_f=20[mA]$.

Beregner spenningsfall over seriemotstanden.
\[U_R=U_{kilde}-U_{LED}= 9-2=7[V]\]

Beregner motstandsverdien
\[R=\frac{U_R}{I_f}=\frac{7}{20 \cdot 10^{-3}}\]

\end{solution}








\vspace{0.5cm} % Add space after the solution

\begin{question}[name=Oppgave, topic=dioder]
En likeretterdiode har et spenningsfall på $0,7 [V]$ over seg i lederetning. Hvor stor effekt omsettes det i dioden når strømmen er $2 [A]$?
\end{question}

\vspace{0.5cm} % Add space after the solution


\begin{solution}[name=Løsningsforslag oppgave]
\[P=U\cdot I = 0,7\cdot2=1,4 [W]\]

\end{solution}

\vspace{0.5cm} % Add space after the solution

\begin{question}[name=Oppgave, topic=dioder]
	Hvilken av kretsene vist i Figur \ref{fig:3kretser} er koblet slik at dioden står i lederetning?

	\begin{figure}[H]
		\centering
		\includegraphics[width=0.9\textwidth]{diode/figurer/3Kretser.png}
		\caption{Tre forskjellige diodekretser}
		\label{fig:3kretser}
	\end{figure}

\end{question}

\vspace{0.5cm} % Add space after the solution

\begin{solution}[name=Løsningsforslag oppgave]
	Krets a) og c)
\end{solution}
\vspace{0.5cm} % Add space after the solution

\begin{question}[name=Oppgave, topic=dioder]
Hvilken spenning vil man måle mellom punkt1 og punkt2 i Figur \ref{fig:10Dserie}.

	\begin{figure}[H]
		\centering
		\includegraphics[width=0.7\textwidth]{diode/figurer/10SerieD.png}
		\caption{Krets med 10 dioder i serie}
		\label{fig:10Dserie}
	\end{figure}

\end{question}

\vspace{0.5cm} % Add space after the solution

\begin{solution}[name=Løsningsforslag oppgave]
	Summerer opp alle spenningsfallene i kretsen.
	\begin{multline*}
	U_{D_{tot}}=\sum_{i=1}^{10} U_{D_i}=\sum_{i=1}^{10} 0,7 \Rightarrow \\	U_{D1}+U_{D2}+U_{D3}+U_{D4}+U_{D5}+U_{D6}+U_{D7}+U_{D8}+U_{D9}+U_{D10} \Rightarrow \\ U_{D_{tot}}=10\cdot0,7=7 [V]
		%SPICEmodell - DC-R10D.asc
	\end{multline*}
\end{solution}

\vspace{0.5cm} % Add space after the solution


\begin{question}[name=Oppgave, topic=dioder]
Hvilken spenning vil man måle mellom punkt1 og punkt2 i Figur \ref{fig:DserieParaMix}

	\begin{figure}[H]
	\centering
	\includegraphics[width=0.7\textwidth]{diode/figurer/6serieOgParaD.png}
	\caption{Krets med dioder i serie og parallell}
	\label{fig:DserieParaMix}
	\end{figure}


\end{question}

\vspace{0.5cm} % Add space after the solution

\begin{solution}[name=Løsningsforslag oppgave]
Siden spenningsfallet over grenene $D1 \rightarrow D4$ er lik grenen $D5 \rightarrow D8$ kan man summere spenningsfallet over en av de for å finne spenningsfallet frem til anoden av $D9$.
\[U_{D1-D4}=U_{D1}+U_{D2}+U_{D3}+U_{D4}= 4\cdot0,7=2,8[V]\]

Benytter det beregnede spenningsfallet og legger til spenningsfallet over $U_{D9}$ og $U_{D10}$.

\[U_{D_{tot}}=U_{D1-D4}+U_{D9}+U_{D10}=2,8+0,7+0,7=4,2 [V]\]
%SPICEmodell - DC-R8D2S.asc

\end{solution}

\vspace{0.5cm} % Add space after the solution



\begin{question}[name=Oppgave, topic=dioder]
	Hvilke lamper lyser i Figur \ref{fig:hvilkenLampe}?

	\begin{figure}[H]
		\centering
		\includegraphics[width=0.5\textwidth]{diode/figurer/hvalyser.png}
		\caption{Lampekrets}
		\label{fig:hvilkenLampe}
	\end{figure}


\end{question}

\vspace{0.5cm} % Add space after the solution

\begin{solution}[name=Løsningsforslag oppgave]
Med utgangspunkt i spenningskildens polaritet så vil lampe 1 og lampe 2 lyse.

\end{solution}

\vspace{0.5cm} % Add space after the solution


\begin{question}[name=Oppgave, topic=dioder]
Din elektriske sportsbil får tilført en likespenning fra en hurtiglader som vist i Figur \ref{fig:diodeLading}. Hva skjer dersom likespenningen fra spenningskilden blir koblet til med feil polaritet?

	\begin{figure}[H]
		\centering
		\includegraphics[width=0.7\textwidth]{diode/figurer/DiodeLading.png}
		\caption{Enkel ladekrets med sikring}
		\label{fig:diodeLading}
	\end{figure}

\end{question}

\vspace{0.5cm} % Add space after the solution

\begin{solution}[name=Løsningsforslag oppgave]

Under vanlige driftsforhold og korrekt polaritet så vil dioden stå i sperreretning og det vil ikke bevege seg strøm gjennom den. Dersom man kobler feil polaritet som beskrevet i oppgaven så vil dioden befinne seg i lederetning. Siden dioden har en relativt lav motstand i lederetning, og strømmen naturlig velger veien tilbake til den negativ polaritet med minst motstand, som vil strømmen bevege seg gjennom dioden. Strømmen vil være opp mot maksimal kortslutningsstrøm for kilden og $>>$\footnote{Tegnet betyr mye større enn. Eksempel: $9\cdot10^{9}>>1\cdot10^{-10}$} enn nominell strøm. Det igjen vil føre til at sikringen løser. Et eksempel er vist i \ref{fig:diodeLadingSol}.
	\begin{figure}[H]
	\centering
	\includegraphics[width=0.7\textwidth]{diode/figurer/DiodeLading–SOL.png}
	\caption{Løsning på enkel ladekrets med sikring}
	\label{fig:diodeLadingSol}
\end{figure}


\end{solution}




\begin{question}[name=Oppgave, topic=dioder]
	Din elektriske sportsbil får tilført en likespenning fra en nå oppgradert hurtiglader sammenlignet med løsningen vist i Figur \ref{fig:diodeLading}. Hva skjer nå dersom spenningskilden blir koblet med feil polaritet på den nye laderen som vist i Figur \ref{fig:diodeLadingG}?

	\begin{figure}[H]
		\centering
		\includegraphics[width=0.7\textwidth]{diode/figurer/GretzLading.png}
		\caption{Ladekrets med brolikeretter}
		\label{fig:diodeLadingG}
	\end{figure}

\end{question}

\vspace{0.5cm} % Add space after the solution

\begin{solution}[name=Løsningsforslag oppgave]
Det skjer ingenting siden bro-likeretteren snur polariteten og sørger for riktig polaritet til bilen. I Figur \ref{fig:BroLadingSol} kan man observere retningen på strømmen under nominelle forhold, og i Figur \ref{fig:BroLadingSol2} kan man observere hva som skjer dersom man bytter polaritet.

\begin{figure}[H]
	\centering
	\includegraphics[width=0.7\textwidth]{diode/figurer/GretzLadingVanlig-SOL.png}
	\caption{Løsning på ladekrets under nominelle forhold}
	\label{fig:BroLadingSol}
\end{figure}

\begin{figure}[H]
	\centering
	\includegraphics[width=0.7\textwidth]{diode/figurer/GretzLadingFeil-SOL.png}
	\caption{Løsning på ladekrets med feil polaritet}
	\label{fig:BroLadingSol2}
\end{figure}


\end{solution}





\vspace{0.5cm} % Add space after the solution

%DUPLIKAT
\begin{comment}
	innhold...

\begin{question}[name=Oppgave, topic=dioder]
	Hvilken av kretsene vist i \ref{fig:3kretser} er koblet slik at dioden står i lederetning?

	\begin{figure}[H]
		\centering
		\includegraphics[width=0.9\textwidth]{diode/figurer/3Kretser.png}
		\caption{Tre forskjellige diodekretser}
		\label{fig:3kretser}
	\end{figure}

\end{question}

\vspace{0.5cm} % Add space after the solution

\begin{solution}[name=Løsningsforslag oppgave]
	Krets a) og c)
\end{solution}
\vspace{0.5cm} % Add space after the solution

\end{comment}


\begin{question}[name=Oppgave, topic=dioder]
Beregn strømmen i kretsen som er vist i Figur \ref{fig:2kilder}.
	\begin{figure}[H]
			\centering
			\includegraphics[width=0.7\textwidth]{diode/figurer/kretsM2kilder.png}
			\caption{Diodekrets med to kilder}
			\label{fig:2kilder}
		\end{figure}

\end{question}

\vspace{0.5cm} % Add space after the solution

\begin{solution}[name=Løsningsforslag oppgave]
Diode $D1$ er koblet i sperreretning og kan betraktes som brudd. Strømmen gjennom $R_1$ beregnes ut fra kretsens spenningsfall.

\[I_{tot}=\frac{U_1-U_2-U_{diode}}{R_1}=\frac{24-4-0,7}{2\cdot10^3}=9,65 [mA]\]

\end{solution}
\vspace{0.5cm} % Add space after the solution


\begin{question}[name=Oppgave, topic=dioder]
Gitt det påtrykte signalet vist i Figur \ref{fig:20Sine}, beregn den maksimale strømmen og spenning for kretsen vist i Figur \ref{fig:acKrets}.
\begin{figure}[H]

	\begin{minipage}[c]{0.45\linewidth}
		\includegraphics[width=\linewidth]{diode/plot/20Sinus.png}
		\caption{Signal på inngangen}
		\label{fig:20Sine}
	\end{minipage}
	\hfill
	\begin{minipage}[c]{0.45\linewidth}
		\includegraphics[width=\linewidth]{diode/figurer/diodeSinus.png}
		\caption{Enkel diodekrets}
		\label{fig:acKrets}
	\end{minipage}
\end{figure}

\end{question}

\vspace{0.5cm} % Add space after the solution

\begin{solution}[name=Løsningsforslag oppgave]
Beregner maksimale spenningen $U_{ut_{peak}}$
\[U_{ut_{peak}}=U_{inn_{peak}}-U_{diode_{peak}}=20-0,7=19,3 [V]\]

Finner så den høyeste strømmen i kretsen  $I_{peak}$
\[I_{peak}=\frac{U_{ut_{peak}}}{R}=\frac{19,3}{500}=38,6 [mA]\]

\end{solution}
\vspace{0.5cm} % Add space after the solution






\begin{question}[name=Oppgave, topic=dioder]
Tegn hvordan spenningen over dioden i Figur \ref{fig:enklAcDiode} endrer seg med tiden for to hele perioder.
	\begin{figure}[H]
	\centering
	\includegraphics[width=0.5\textwidth]{diode/figurer/ACdiode.png}
	\caption{Enkel diodekrets}
	\label{fig:enklAcDiode}
\end{figure}

\end{question}

\vspace{0.5cm} % Add space after the solution

\begin{solution}[name=Løsningsforslag oppgave]
	\begin{figure}[H]
	\centering
	\includegraphics[width=0.5\textwidth]{diode/figurer/ACdiode-LOS.png}
	\caption{Løsning på enkel diodekrets}
	\label{fig:enklAcDiodeLOS}
\end{figure}

\end{solution}
\vspace{0.5cm} % Add space after the solution