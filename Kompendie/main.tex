\documentclass[12pt]{report}

\usepackage[norsk]{babel}
\usepackage{graphicx} % Required for inserting images
\usepackage{graphicx, array, carlito, fancyhdr, comment, etoc, lipsum, enumitem, standalone, subfiles, newclude, float, enumitem, amsmath, hyperref, adjustbox, titlesec, pdfpages}

\usepackage[style=ieee, sorting=none]{biblatex}

% Fjerne Teksten Kapittel fra overskrift. Justerer plasseringen av chapter navn
\titleformat{\chapter}[display]
{\Huge\bfseries}
{}
{-120pt}
{\thechapter.\ }
\titleformat{name=\chapter,numberless}[display]
{\Huge\bfseries}
{}
{-120pt}
{}
\titlespacing*{\chapter}{0pt}{70pt}{10pt} % Juster verdien 40pt for å endre avstanden


%Gir automatisk nytt avsnitt ved linjeskift
\usepackage[parfill]{parskip}

%Ikke inrykk ved nytt avsnitt
\setlength{\parindent}{0pt}

\usepackage{exsheets}
%Vis løsningsforslag på alle oppgaver
\SetupExSheets{solution/print=false}


\usepackage[a4paper, total={6in, 8in}]{geometry}



% Laste inn referanser
\addbibresource{backmatter/referanser.bib}

% Adding abstract funciton
\newcommand{\chapabstract}[1]{
	\begin{quote}
		\singlespacing\small
		\rule{14cm}{1pt}
		#1
		\vskip-4mm
		\rule{14cm}{1pt}
\end{quote}}


% Set up the fancyhdr package
\pagestyle{fancy}
\fancyhf{} % Clear all header and footer fields

% Adjust the footer position
\setlength{\footskip}{3cm} % You can adjust this value as needed

% Define the footer
%\fancyfoot[L]{\vspace{1cm} \makebox[0pt][l]{\hspace{0.5cm} Elektroniske systemer}} % Left-aligned course name, lifted up
\fancyfoot[R]{%
	\begin{minipage}{2cm}
		\centering
		\includegraphics[height=1cm]{frontmatter/bilder/r171JViZ6gmf8WXujFQw.jpg}\\
%		\vspace{-0.5cm} % Adjust vertical space to center page number
%		\thepage
	\end{minipage}
}
%\fancyfoot[C]{\thepage}

\fancyhead{}
\fancyhead[LE,RO]{\nouppercase{\rightmark}}
\fancyhead[RE,LO]{Side \thepage}





% ------------------    HER STARTER DOKUMENTET   ------------------

\begin{document}
%Setter inn tittelsiden som eksternt dokument
\begin{titlepage}
    \begin{center}
    \vspace*{1cm}

    \Huge
	Elektroniske systemer

    \Large
    \vspace{0.5cm}
    Øvingsoppgaver for analoge komponenter og måleteknikk i emnet elektroniske systemer


   \vspace{1.5cm}
   \textbf{Carl Magnus Bøe}

    \today{}
	\vfill




    \includegraphics[width=0.5\textwidth]{frontmatter/bilder/r171JViZ6gmf8WXujFQw.jpg}

    \Large
    Fagskolen Viken\\
    01TE00\textit{X} EITKELFH\textit{XX} - Emne 5\\
    Fredrikstad\\

    \end{center}
\end{titlepage}

%Innholdsfortegnelse
\tableofcontents
\chapter{Introduksjon}
\label{ch:introduksjon}
Dette kapittelet inneholder generell informasjon om kompendiet med bakgrunn for arbeidet og oppbygging av dokumentet.

\subsection{Bakgrunnsinformasjon}
Dette dokumentet er et kompendiet som inneholder øvingsoppgaver relevante til første delen av emnet elektroniske systemer. Siden dokumentet blir kontinuerlig revidert er det datoen på forsiden som angir versjonen av dokumentet. Alle oppgavene er med dette dokumentet samlet på ett sted, sammen med løsningsforslag til alle oppgaver. Når man jobber seg gjennom oppgavene så anbefales det at man også gjør simuleringer. Noen anbefalte simuleringsverktøy er omtalt i Kapittel \ref{ch:programvare}.

Dersom du har kommentarer, forslag til oppgaver eller funnet noe som er feil vennligst send en epost til \href{mailto:carlbo@afk.no}{carlbo@afk.no}.

\subsection{Oppbygning av kompendiet}
Kompendiet er delt opp i hovedgrupper hvor undergrupper som forskjellige komponenter er beskrevet som seksjoner. For hver seksjon presenteres først alle oppgavene, før løsningsforslaget blir presentert i slutten av den samme seksjon som oppgavene.
\chapter{Programvare}
\label{ch:programvare}
Dette kapittelet omtaler forskjellig programvare relevant i emnet.


\section{Simulering}
\subsection{CircuitMaker2000}
\href{https://winworldpc.com/product/circuitmaker/2000}{CircuitMaker2000}
\subsection{LTspice}
\href{https://www.analog.com/en/resources/design-tools-and-calculators/ltspice-simulator.html}{LTspice}
\subsection{OpenModelica}
\href{https://openmodelica.org/}{OpenModelica}

\section{Tegne kretser}
\subsection{Draw.io}


\chapter{Analoge komponenter}

\section{Dioder}


Dette kapittelet inneholder oppgaver relatert til halvleder dioder. Om ingenting annet er gitt i oppgaven så antar vi et ideelt spenningsfall over dioden på $0,7[V]$.\\
\subsection{Lysdioder - LED}
Lysdioder har forskjellige spenningsfall for samme farge avhengig av modellserie og produsent. For optimal verdi må man lese databladet til dioden. 

Tabell \ref{tab:LEDspenningsfall} viser et generelt spenn av verdier.

\begin{table}[!ht]
	\caption{Spenningsfall for forskjellige lysdioder}
	\label{tab:LEDspenningsfall}
	\begin{center}
		\begin{tabular}{|l|c|c|} 
			\hline
			Farge & Spenningsfall & Enhet \\ [0.5ex] 
			\hline\hline
			Hvit & $3,0 - 5,0$ &$ [V]$ \\
			\hline
			Fiolett: &  $2,8 - 4,0$ &$ [V]$\\
			\hline
			Blå: &  $2,5 - 3,7$ &$ [V]$\\
			\hline
			Grønn: &  $1,6 - 4,0$ &$ [V]$\\
			\hline
			Gul: &  $2,0 - 2,4$ &$ [V]$\\
			\hline
			Oransje: &  $2,0 - 2,1$ &$ [V]$\\
			\hline
			Rød: &  $1,5 - 2,0$ &$ [V]$\\
			\hline
			Infrarød: &  $1,2 - 1,9$ &$ [V]$\\
			\hline
		\end{tabular}
	\end{center}
\end{table}



%Setter inn oppgaver fra ekstern ark
\subsection{Oppgaver}
\subsubsection{Dioder}
\input{diode/oppgaver_dioder}
\subsubsection{Zenerdioder}
\input{diode/oppgaver_zenerdioder}

\subsection{Løsningsforslag}
\printsolutions[section]



\newpage

\section{Tyristor, triac og diac}
\begin{question}[name=Spørsmål, topic=tyristor]
	Tegn symbolene for følgende komponenter.
	\begin{enumerate}[label=\roman*]
		\item Halvlederdiode
		\item LED
		\item Zenerdiode
	\end{enumerate}
\end{question}


\begin{solution}[name=Løsningsforslag]
tatata

%	\begin{figure}[H]
%		\centering
%		\includegraphics[width=0.7\textwidth]{diode/figurer/oppgave1.png}
%		\caption{Eksempel på forskjellige diode symboler.}
%		\label{fig:diodeSeeeymb}
%	\end{figure}
\end{solution}
\printsolutions[section]

\section{BJT transisor}

\section{FET transistor}

\section{Forsterker i praksiss}

\section{Måleteknikk}

\newpage

\printbibliography%[title={Referanser}]

\appendix
\chapter[LED Datasheet]{LED Datasheet}
\label{paper-a}

\emph{Datablad fra en standard LED \cite{LEDdatasheet}.}
	
	\bigskip

\includepdf[pages=-]{backmatter/pdfVedlegg/1LED_datasheet1498852.pdf}


\end{document}
