\chapter{Introduksjon}
\label{ch:introduksjon}
Dette kapittelet inneholder generell informasjon om kompendiet, bakgrunn for arbeidet, oppbygging av dokumentet og lisensinformasjon.

\section{Bakgrunnsinformasjon}
Dette dokumentet er et kompendium som inneholder øvingsoppgaver relevante for delen analoge komponenter i emnet elektroniske systemer. Siden dokumentet blir kontinuerlig revidert, er det datomerkingen på forsiden som angir versjonen av dokumentet. Målet med dette dokumentet er å samle alle øvingsoppgaver sammen med løsningsforslagene i ett dokument.

Når man jobber med oppgavene, anbefales det at man også gjør simuleringer. %Noen anbefalte simuleringsverktøy er omtalt i Kapittel \ref{ch:programvare}.

Dersom du har kommentarer, forslag til oppgaver eller har funnet noe som er feil, vennligst send en epost til \href{mailto:carlbo@afk.no}{carlbo@afk.no}.

\section{Oppbygning av kompendiet}
Kompendiet er delt opp i hovedgrupper, hvor undergrupper som for eksempel forskjellige elektroniske komponenter er beskrevet som seksjoner. For hver seksjon presenteres først alle oppgavene, før løsningsforslaget blir presentert i neste seksjon.

\begin{comment}
	\section{Lisensinformasjon}
	Dette dokumentet er basert på materiale fra \textit{Lessons In Electric Circuits} av Tony R. Kuphaldt, distribuert under Design Science License. Originaldokumentet kan finnes på \href{https://www.ibiblio.org/kuphaldt/electricCircuits/}{Lessons In Electric Circuits}. Eventuelle modifikasjoner og avledede verk er også distribuert under Design Science License.
	
	Deler av dette kompendiet er utviklet av Carl Magnus Bøe. Disse delene inkluderer:
	\begin{enumerate}[label=\roman*)]
		\item Kapittel \ref{sec:diodeOppgave} / \ref{sec:diodeLøsning}
		\item 
	\end{enumerate}
\end{comment}