\documentclass{book}
\begin{document}


\chapter{Calculus}

Matematikk er studiet av en kunstig verden, befolket av abstrakte enheter og strenge regler som styrer disse enhetene. Matematikere som er dedikert til ren matematikk, har stor respekt for disse reglene, da integriteten til deres kunstige verden avhenger av dem. For å bevare denne integriteten, må deres kollektive arbeid være \textit{strengt}, og aldri tillate slurvete håndtering av reglene eller intuitive sprang som ikke er bevist.

Mange av verktøyene og teknikkene utviklet av matematikere er imidlertid svært nyttige for å forstå den virkelige verden. Når matematiske regler anvendes på virkelige fenomener, benyttes det ofte en mer pragmatisk tilnærming enn det matematikere ville foretrekke.

Spenningsforholdet mellom rene matematikere og de som anvender matematikk på virkelige problemer, ligner forholdet mellom lingvister og de som bruker språk i hverdagen. Alle språk har regler, men de fleste mennesker bruker språk mer fleksibelt for å beskrive og forstå verden rundt seg. For puristen er dette støtende; for pragmatikeren er det praktisk.

Matematikk kan sammenlignes med et språk. Jo mer du forstår matematikk, jo større "ordforråd" vil du ha for å beskrive prinsipper og fenomener i verden rundt deg. Ferdigheter i matematikk gir også muligheten til å forstå forhold mellom forskjellige ting, noe som er et kraftig verktøy for å lære nye konsepter.

Denne boken er ikke skrevet for matematikere, men for de som ønsker å forstå industriell prosessmåling og kontroll. Dette kapittelet dekker visse matematiske konsepter på en pragmatisk måte, med det formål å anvende disse konseptene på virkelige systemer.

\vskip 10pt

Matematikere, dekk øynene for resten av dette kapittelet!


\end{document}

