\subsection{Absolutt Usikkerhet}

Absolutt usikkerhet refererer til den faktiske mengden usikkerhet i en måling, uttrykt i samme enhet som selve målingen. For eksempel, hvis du måler lengden av et bord til 2,00 meter med en usikkerhet på ±0,01 meter, er den absolutte usikkerheten 0,01 meter. Denne typen usikkerhet brukes ofte når man ønsker å vite den eksakte mengden usikkerhet i en måling.

\subsection{Relativ Usikkerhet}

Relativ usikkerhet er forholdet mellom den absolutte usikkerheten og selve målingen, og den uttrykkes ofte som en prosentandel. Formelen for relativ usikkerhet er:

\begin{equation}
	\label{eqn:relUsikkerhet}
		 	Relativ\ usikkerhet = \frac{Absolutt\ usikkerhet}{M \text{å}lt verdi} \cdot 100 \%
\end{equation}



For eksempel, hvis du har en måling på 2,00 meter med en absolutt usikkerhet på ±0,01 meter, er den relative usikkerheten $\frac{0,01}{2,00} \cdot 100 \% = 0,5\%$. Relativ usikkerhet brukes ofte for å sammenligne usikkerheten i forskjellige målinger eller for å vurdere nøyaktigheten av en måling i forhold til størrelsen på målingen.